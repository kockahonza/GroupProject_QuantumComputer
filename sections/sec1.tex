\section{Introduction}
\begin{itemize}
    \item brief overview of what a quantum computer is
    \item brief timeline/history of quantum computers
    \item where does this report fit into current literature
\end{itemize}
\subsection{Why build a quantum computer?}
\begin{itemize}
    \item RSA encryption
    \item How QC is better at breaking RSA encryption (explain simultaneous states)
    \item Where else can the benefits be used (other than in cryptography, e.g. bio simulation)
\end{itemize}
\subsection{How to build a quantum computer}
This is basically just acknowledging there's different ways of doing it, mention where Google, IBM and IonQ are at. Then say we've chose Trapped Ion and explain structure of report 
\begin{itemize}
    \item multiple ways of doing it
    \item give some historical developments on the different technologies
    \item who are the key players in building quantum computers
    \item briefly explain how the report will be structured, why have we chosen it this way
\end{itemize}

The purpose of this report is to describe the physical realisation of a quantum computer. 
It will give a brief explanation of the quantum mechanical concepts which are key to quantum computing such as entanglement and superposition. 
The theory of storing and processing of information using quantum mechanical systems is described broadly before a detailed insight into `building' a quantum computer.
Quantum computing is a very active field of research with numerous methods of implementation being studied including photonic and superconducting systems. 

Quantum supremacy is a term used to describe when a quantum computer is able to complete a calculation that a classical computer could not achieve in a reasonable amount of time. 
Google was first to claim quantum supremacy in 2019 with their 53 bit `Sycamore' quantum computer which reportedly solved a problem in under 5 hours which would take a classical computer over 10 000 years. \cite{gibney_hello_2019} 
However these claims were disputed by IBM - another large company heavily invested in the quantum computing race.
The largest quantum computer to date is the IBM osprey with 433 qubits announced in November 2022, both the Osprey and Sycamore computers use superconducting qubits (where qubits are the quantum realisation of classical computer bits).

This report will focus on trapped ion systems where in [date] [company] announced a [nbit] computer. Although trapped ions have not yet reached the number of qubits seen from superconducting systems there are many advantages to them, some of these are: 
\begin{enumerate}
    \item The cost of ion trap systems is relatively low - therefore research can be carried out by academics and be peer-reviewed. This affects the amount of research materials available on the public domain. As opposed to superconducting systems which require vast investments leading to only a select few companies able to reseach it and they do not publish much reseach due to their stategic ad
\end{enumerate}