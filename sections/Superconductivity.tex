\section{Superconducting QC stuff}

\begin{itemize}
    \item Currently used by Google, IBM, etc. and other big companies
    \item Cooper pairs are formed when an electron moves through a lattice, it pulls positive ions in the lattice closer to it as it moves along, creating a 'ripple' effect of increasing positive charge distribution, which then attracts nearby electrons to the ripple and forms a couple with the first electron. These electrons are paired and are called Cooper pairs. Their mutual repulsion keeps them away from each other and propels one when the other gets too close, and the other is pulled to the first when the first one creates a denser positive charge distribution. This way they propel each other without force and form superconducting electrons. 
    \item 3 types – Flux, Charge, Phase
    \item Most helpful \url{https://pennylane.ai/qml/demos/tutorial_sc_qubits.html}
    \item DVC1 : Very scalable because Josephson junctions are made on chips with 'easy' manufacturing techniques, however since the principle is having quantum effects on macroscopic level, too much scaling has issues with losing some quantum properties.
    \item DVC2 : Easily satisfied, because "Since the excited states in an artificial atom are short-lived and prefer to be on the ground state, all we have to do is wait for a short period. If the circuit is well isolated, it is guaranteed that all the qubits will be in the ground state with a high probability after this short interval." from PennyLane
    \item DVC3 : Hard because qubits are short-lived
    \item DVC4 : Somewhat challenging, currently done using 'capacitative coupling', however WiKipedia Superconducting QC page says doable
    \item DVC5 : easily doable by shining specific frequency light that matches optical cavity transmission frequency and gets scattered by superconducting circuit, thus interacting with the qubit and collapsing it to one state
    \item 
    \url{https://jonathan-hui.medium.com/qc-how-to-build-a-quantum-computer-with-superconducting-circuit-4c30b1b296cd}
    \item 
    \url{https://www.reddit.com/r/askscience/comments/h0t51/comment/c1rr48v/?utm_source=share&utm_medium=web2x&context=3}
    \item \url{https://quantum.phys.cmu.edu/QCQI/QC_CMU2}
    \item \url{https://www.nature.com/articles/s41586-019-1666-5}
    \item \url{https://www.nature.com/articles/nature07128#Sec1}
    \item \url{https://www.nature.com/articles/s41534-016-0004-0#Sec2}

\end{itemize}