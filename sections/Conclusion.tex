\section{Conclusion}

Recap motivation and introduction

Recap trapped ion

The report introduced the multiple approaches to quantum computing using trapped ion systems and how they have currently been implemented. 
It was described how the five DiVencenzo criteria could be met from this approach with the acknowledgement that the first objective has only been met in part - there is well defined qubits but scalability is challenging.
This is because the qubits are ions which are all in the same trap to achieve entanglement, however increasing the numbers of ions in a single trap becomes increasing difficult and is likely to be a limiting factor for this approach. 

There are multiple methods of ion trapping and the report has described in detail Paul traps however it was noted that Penning traps are also used for quantum computing.
Penning traps use both magnetic and elecgtric fields to trap ions while Paul traps use a time varying electric field and the Paul trapped was focued on Paul traps as they are easier to manipulate sin

Recap photonic

Give some comments looking to the future