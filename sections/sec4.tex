\section{Michael's Section}
Finding a physical representation of qubits that fulfil all the mathematical requirements is not easy.
Gaining control over a single quantum system has been a want since about 1970. word this so better
It is possible to see quantum effects on a vast number of combined quantum systems such as in superconductors. \cite{nielsen_quantum_2010}
Or in particle collisions again quantum effects are observed however there isn't control over a single quantum system.
Qubits are the physical realisation of controlling a single quantum system and its state.
You require a system with many degrees of freedom that can encode quantum information (qubits). \cite{bergou_quantum_2021}

\subsection{Optical Photon Quibit}

{\bf Why Photons?}
\begin{itemize}
    \item Photons are massless, chargeless and don't interact with each other or other mass much. \cite{nielsen_quantum_2010}
    \item Can be guided long distances without much energy loss using optical fibers. \cite{nielsen_quantum_2010}
    \item Quantum information can be transmitted over long distances using photons as demonstrated by quantum entanglement over 1200km. \cite{yin_satellite-based_2017}
    \item photons can maintain entanglement over long distances and time period allowing the transmission quantum information. - this supper secure wow \cite{thibault_team_nodate}
\end{itemize}

\vspace{1em}
{\bf How photons?}

\begin{itemize}
    \item The transverse polarisation state of a photon can be used to represent qubits. 
    These are vertical, horizontal linear and left, right circular polarization. \cite{bergou_quantum_2021}
    
    \item These states can also be maintained - in isotropic materials photons polarization does not change as it propagates. \cite{bergou_quantum_2021}
    
    \item $\left\vert 0 \right\rangle = \left\vert H\right\rangle $ and $\left\vert 1 \right\rangle = \left\vert V\right\rangle $ where H and V are horizontal and vertical polarisation. \cite{bergou_quantum_2021}

\end{itemize}
\vspace{1em}
{\bf Need a single photon source}
\begin{itemize}
    \item lasers with very low output can emit single electrons... but 90$\%$ of the time no photon is emitted and when there is a photon emitted 5$\%$ of than one is emitted.\cite{nielsen_quantum_2010}
    \begin{itemize}
        \item so lasers CANNOT be used - we need sources that can be synchronised not possible if you don't know when a photon is actual coming out
    \end{itemize}
\end{itemize}
\vspace{1em}
{\bf Quantum Dot}
\begin{itemize}
    \item Quantum dots can (kinda) be a single photon source
    \item from wiki quantum dots seem to be `semiconducting particles a few nanometers in size' therefor shining light on it would emit one photon is the idea
    \item need to understand $T_1$ and $T_2$ coherance times.... think $T_2$ is the time taken for the inital and final states to be uncorrelated (i.e. $|1\rangle \rightarrow \alpha |0\rangle + \beta |1\rangle$ a liner combination of them both?
    \item anyway... using a narrow-linewidth laser which had the same resonance as the quantum dot single photon emission was achieved with a $T_2=22ns$ (apparently long) was achieved. \cite{lodahl_interfacing_2015}
\end{itemize}
\vspace{1em}
{\bf DiVincenzo Criteria} \cite{bergou_quantum_2021}
\begin{itemize}
    \item Scalability with well defined qubits
    
    \begin{itemize}
        \item $\left\vert 0 \right\rangle = \left\vert H\right\rangle $ and $\left\vert 1 \right\rangle = \left\vert V\right\rangle $ where H and V are horizontal and vertical polarisation.  
        \item polarisation state can be maintained for a long time
        \item or timebin/path encoding \cite{obrien_optical_2007}

        \item eventually need like $10^{11}$ single photon states. posible way of doing this is using $10^11$ sources that can reliably produce a photon as in when requested
        \begin{itemize}
            \item don't exist yet, but are being worked on some promising are quantum dots, trapped ions and atoms, color centers in diamonds, semiconductors \cite{slussarenko_photonic_2019}
        \end{itemize}
    \end{itemize}

    \item Initializing qubits to a simple fiducial state
    \begin{itemize}
        \item quantum dot - single photon source. when excited on resonance by low laser prob of emmiting two photons instead of one is super low \cite{santori_indistinguishable_2002}
    \end{itemize}

    \item  A qubit-specific measurement capability
    \begin{itemize}
        \item an ideal photon detector which detects every individual photon with no dark count (false positive) doesn't yet exist. 
        \item one of the most used detectors currently is the Si avalanche photodiode... only has a detection efficiency of 65$\%$! so the probability of detecting 10 photons with 10 detectors is less than $2\%$. \cite{slussarenko_photonic_2019}
        \item better now we have superconducting nanowire single-photon detectors (SNSPDs) with reset rate $\sim$40ns and detection efficiency $>95\%$ ... but these need temos of around 2K. \cite{santori_indistinguishable_2002}
        
    \end{itemize}
    
    \item Long relevant decoherence times
    \begin{itemize}
        \item 
    \end{itemize}

    \item A ``universal'' set of quantum gates
    \begin{itemize}
        \item birefringent wave plates for one qubit gates \cite{obrien_optical_2007}
    
        \item deeterministic CNOT gates difficult (these are the double qubit ones). A 2007 paper suggests that for a near deterministic CNOT gate (i.e. not deterministic just low prob) it requires $>$10 000 entangled photons for only $>95\%$ success probability \cite{obrien_optical_2007}
    \end{itemize}

    \item The ability to interconvert stationary and flying qubits
    \begin{itemize}
        \item can use polarizing beam splitters to convert between polarisation and path encoding \cite{obrien_optical_2007}
    \end{itemize}

    \item The ability to transmit flying qubits between specified locations
    \begin{itemize}
        \item well photons are transmitted all the time!
        \item actually kind of difficult cause photons are always 'flying' so some kind of optical quantum memory may be necesarry to store them. 
    \end{itemize}
\end{itemize}
