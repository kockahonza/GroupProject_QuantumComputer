\section{Michael's Section}
Finding a physical representation of qubits that fulfil all the mathematical requirements is not easy.
Gaining control over a single quantum system has been a want since about 1970. word this so better
It is possible to see quantum effects on a vast number of combined quantum systems such as in superconductors. \cite{nielsen_quantum_2010}
Or in particle collisions again quantum effects are observed however there isn't control over a single quantum system.
Qubits are the physical realisation of controlling a single quantum system and its state.
You require a system with many degrees of freedom that can encode quantum information (qubits). \cite{bergou_quantum_2021}

\subsection{Optical Photon Quibit}

Why Photons?
\begin{itemize}
    \item Photons are massless, chargeless and don't interact with each other or other mass much. \cite{nielsen_quantum_2010}
    \item Can be guided long distances without much energy loss using optical fibers. \cite{nielsen_quantum_2010}
    \item Quantum information can be transmitted over long distances using photons as demonstrated by quantum entanglement over 1200km. \cite{yin_satellite-based_2017}
    \item photons can maintain entanglement over long distances and time period allowing the transmission quantum information. - this supper secure wow \cite{thibault_team_nodate}
\end{itemize}


How photons?

\begin{itemize}
    \item The transverse polarisation state of a photon can be used to represent qubits. 
    These are vertical, horizontal linear and left, right circular polarization. \cite{bergou_quantum_2021}
    \item These states can also be maintained - in isotropic materials photons polarization does not change as it propagates. \cite{bergou_quantum_2021}
    \item $\left\langle 0 \right\vert $
\end{itemize}