\section{Jans notes}
Ok, I'm taking over this section for now
\begin{itemize}

\item
For "types" fo QCters I see photonic and trapped ion for now at least.

\item
Trapped ion type seems quite nice, apparently it works well, but they haven't really scaled to hundreds of qubits.


\item
Found this thing called DiVincezo criteria which might be nice to mention or something, it's 5 criteria that need to be met for a given
hypothetical QCter type/implementation:
\begin{itemize}
    \item A physical system containing well-defined 2 level quantum systems (qubits) which can be isolated from the environment.
    \item The ability to initialize this system in a well defined, determinate state.
    \item A set of universtal quantum gates which can be applied to each qubit or possibly pairs (or more) of them.
    \item Qubit decoherence times much greater than times for quantum gates to be applied.
    \item The ability to read out qubit state with high accuracy.
    \item ** In addition 2 more were mentioned by him, the ability to convert between "stationary" (like trapped ions) and "flying" (photons) states which would be necessary for a quantum network.
\end{itemize}

\item Trapped ion systems, kinda taken from Bruzewicz 2019, they also mention the DiVincezo stuff
\begin{itemize}
    \item The actual states are the electronic states of some atom/ion.
    \item A couple types: hyperfine, Zeeman, fine structure and optical the differences between them is the particular energy levels that are used for states.
            The names make it decently clear, fine structure means the fine structure split states are used, optical means states which are transitioned between by photon emission/absorption.
    \item One of the things to deal with is that the atoms energies come from both their structure and their motion within the trap (which is also quantized).
    \item Some of the manipulations take place through lasers.
    \item Main pros are long coherence time and great gate and initialization/readout fidelity, drawbacks include speed, the gates are quite slow compared to other qubit technologies.
    \item Unordered random facts
    \begin{itemize}
        \item single qubit gates take a couple micro seconds and double about 10-100 micro seconds
        \item coherence times range from 0.2 to 600s (600s achieved for hyperfine) and surface code error correction was mentioned, the gate fielity should be good enough for that
        \item with respect to the optional DiVincezo, ions are unlikely to be the "flying" ones even though they can be moved, however high-fielity entanglement between ions and photons has been showed.
        \item most of the main things have been demonstrated by 2004 though the maximum number of ions achieved in a register so far was 20.
    \end{itemize}
\end{itemize}

\end{itemize}


