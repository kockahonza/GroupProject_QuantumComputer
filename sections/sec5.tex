\section{Jans notes}
Ok, I'm taking over this section for now
\begin{itemize}

\item
Found this thing called DiVincezo criteria which might be nice to mention or something, it's 5 criteria that need to be met for a given hypothetical quantuum computer implementation:
\begin{itemize}
    \item A physical system containing well-defined 2 level quantum systems (qubits) which can be isolated from the environment.
    \item The ability to initialize this system in a well defined, determinate state.
    \item A set of universtal quantum gates which can be applied to each qubit or possibly pairs (or more) of them.
    \item Qubit decoherence times much greater than times for quantum gates to be applied.
    \item The ability to read out qubit state with high accuracy.
    \item ** In addition 2 more were mentioned by him, the ability to convert between "stationary" (like trapped ions) and "flying" (photons) states which would be necessary for a quantum network.
\end{itemize}

\item
    The Nielsen book lists quite a few of physical implementations for quantum computers - QHO, optical photon, optical cavity, ion traps and nuclear magnetic resonance

\item Trapped ion systems, kinda taken from Bruzewicz 2019, they also mention the DiVincezo stuff
\begin{itemize}
    \item All the main parts have been demonstrated, the problem is scaling.
    \item The actual states are the electronic states of some atom/ion.
    \item A couple types: hyperfine, Zeeman, fine structure and optical the differences between them is the particular energy levels that are used for states.
            The names make it decently clear, fine structure means the fine structure split states are used, optical means states which are transitioned between by photon emission/absorption.
    \item One of the things to deal with is that the atoms energies come from both their structure and their motion within the trap (which is also quantized).
    \item Some of the manipulations take place through lasers.
    \item Main pros are long coherence time and great gate and initialization/readout fidelity, drawbacks include speed, the gates are quite slow compared to other qubit technologies.
    \item Unordered random facts
    \begin{itemize}
        \item single qubit gates take a couple micro seconds and double about 10-100 micro seconds
        \item coherence times range from 0.2 to 600s (600s achieved for hyperfine) and surface code error correction was mentioned, the gate fielity should be good enough for that
        \item with respect to the optional DiVincezo, ions are unlikely to be the "flying" ones even though they can be moved, however high-fielity entanglement between ions and photons has been showed.
        \item most of the main things have been demonstrated by 2004 though the maximum number of ions achieved in a register so far was 20.
    \end{itemize}

    \item As far as I understand it is in fact a single ion that would represent a given qubit, the obvious way to store them are Penning or Paul traps.
    \item Paul traps rely on time dependent quadrupole fields (oscillating, usually trig or square waveforms).
    \item Penning traps do a mix of E and B fields to get a fairly complicated but alright motion.
    \item As far as I can tell Paul traps are considered the easier ones (and are more developed) but Penning traps have their advantages.

    \item Paul traps
    \begin{itemize}
        \item There are in fact 2 types.
        \item Point Paul traps have oscillating traps in all 3 dimensions, resulting in a single point with net 0 RF force (over a period).
            This can lead to issues when trapping multiple ions in the same trap as they tend to interact/overlap a lot.
        \item Linear Paul traps are the other type, they have an RF quadrupole in 2 dimensions and a static electric potential across the last one, this means the atoms can space better along that line according to that static potential.
        \item Okay, it's actually kinda insanely cool, so nowadays instead of it actually being 4 rods for the quadrupole field and some segments for the static field, they can deform this onto a surface -- as in the trap itself is a little plate with conducting segments to which the voltages are applied.
            Apparently it's still a good enough field and it is easy to make, there is plenty access for lasers and so on.
    \end{itemize}
    \item Atoms are loaded using resonant photo-ionization to make sure only a given isotope is present (in the past bombarding the atom gas with electrons as used and resulted in different isotopes being in the trap)
    \item Typically the excite states are quite high energy and multiple steps might be needed (often using near UV photons) to get them there.
    \item Apparently mostly elements from the 2nd group in the periodic table (or ones with similar properties) are used for trapped ion qubits.
    \item Any 2 states with long lifetimes can be used for the 0 and 1 states and almost always one of them is from the "ground state manifold" and based on where the other is the energy splittings can be anything between MHz to hundreds of THz.
    \begin{itemize}
        \item A big aspect of linear paul trap based systems is how it results in essentially a 1D crystal with the ions all trapped in the same harmonic potential along the axis, this results in shared motional states
        \item As we've just done in ICMP, there can be vibrational modes aka phonons which are "shared" among the ions (sorry don't have a simple but more formal explanation for now).
        \item There are 3N vibrational modes for this sort of a crystal with N ions and each mode is in a harmonic oscillator state (as in can be in a superposition of different harmonic oscillator states).
        \item The article I'm readidng now mentions that "lasers can be used to excite the internal electronic levels dependent upon the ions' vibrational states" which I'm not sure how but I see how that would lead to a way to entangle their states.
    \end{itemize}

    \item So I now skipped to more of the overview/applicability part.
    \item We might want to mention NISQ (Noisy Itentermeiate-Scale Quantum computing) -- it is a term describing the type of quantum computer we might be able to build now or in the very near future.
        It is essentially a subfield, studying what useful applications they could have is done. They require about 100s of qubits and as far as I can tell they "do not handle error correction" by themselves, the algorithms need to take care of that.
    \item A 53 qubit trapped ion system was used for further physics research (as far as I can understand they used it to simulate some complicate quantum many body system, they call is a quantum simulator).
    \item Quantum simulations using quantum computers seems to be a big thing in general, the Nature Physics 2012 Insight collections intro mentions atomic quantum gases, trapped ions, photonic systems and superconducting circuits being demonstrated.
    \item And I have seen a couple mentions of ~10s (up to 20 or so) qubit trapped ion systems though I didn't track them down right now.
\end{itemize}

\end{itemize}


