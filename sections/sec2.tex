\section{Background}
\begin{itemize}
    \item DiVin
\end{itemize}

\subsection{The DiVincenzo Criteria -- to be polished}
The DiVincenzo criteria is a set of objectives that must be met for the physical realisation of a quantum computer. 
It sets out five basic principles which must met to build a functional quantum computer and two more principles for quantum information networks. \cite{bergou_quantum_2021}
This report focuses on the implementation of a single quantum computer and therefore focused on the first five principles.
The principles are:
\begin{enumerate}
    \item Scalability with well defined qubits
    \item The ability to initialise the system in a well defined, determinate state
    \item The ability to read out qubit state with high accuracy
    \item A set of universal quantum gates
    \item Long relevant decoherence times
    \newcounter{enumTemp}
    \setcounter{enumTemp}{\theenumi}
\end{enumerate}
The two principles for the implementation of quantum information networks are:
\begin{enumerate}
    \setcounter{enumi}{\theenumTemp}
    \item The ability to convert between `stationery' and `flying' qubits
    \item The ability to transmit flying qubits between specified locations
\end{enumerate}

There are various physical approaches for quantum computing that satisfy some of the DiVincenzo criteria, however no current approach has successfully satisfied all five.
\vspace{1em}

The first objective to implement well defined qubits which are saliable. An example of a well defined qubit is superposition of vertical and horizontal polarisation of a photons, or the spin up and spin down of electrons. 
There are many methods of creating well defined qubits, however scalability and producing multiple which remain in a superposition state is challenging.
IBM claimed in 2021 to have created the worlds largest quantum processor, the 127 qubit Eagle. \cite{authorfullname_ibm_nodate}
\vspace{1em}

The second objective is to initialise the qubits into a well defined determinate state i.e. initialise the qubit register so all are in the 0 state $\vert 0\rangle \vert 0\rangle$...$\vert 0 \rangle$. \cite{lapierre_divincenzo_2021}
\vspace{1em}

The third objective is to read out qubit state $\vert 0\rangle$ or $\vert 1 \rangle$ with high accuracy. 
This is a challenge with photons as a method for detecting individual photons with no dark count (false positives) does not exist yet.
\vspace{1em}

The fourth objective is to implement a universal set of gates, all operations can be preformed using two quantum gates.
One is gate which flips the state of a single qubit. The other is a CNOT gate which acts on two qubits and flips the state of one qubit only if the 'control' qubit is in state $\vert 1 \rangle$. \cite{lapierre_divincenzo_2021} 
\vspace{1em}

The final objective is for long `relevant' decoherence times, this requires the qubit to stay in a superposition quantum state for longer than the duration of gate operations before the state collapses. 
