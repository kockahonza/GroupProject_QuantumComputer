\section{Mass Spectrometry}
A Mass Spectrometer (MS) is a device used to measure the mass to charge ratio ($\frac{m}{Q}$) of ions.
Usually, this charge to mass ratio is calculated from either how much an ion is deflected in a magnetic field or accelerated in an electric field – ions with the same ratio are deflected/accelerated equally.
This information can be used for a variety of purposes such as determining an unknown isotope in a mixture or to directly measure nuclear masses.