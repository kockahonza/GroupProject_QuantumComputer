\documentclass[a4paper]{article}
\usepackage[top=60pt,bottom=80pt,left=60pt,right=60pt]{geometry}
\usepackage[utf8]{inputenc}
\usepackage[style=phys,url=false,doi=false,isbn=false,eprint=false]{biblatex}
\usepackage{amsmath}
\usepackage{cleveref}
\usepackage{lipsum}
\usepackage{xurl}

\addbibresource{bibliographies/bib1.bib}
\addbibresource{bibliographies/bib2.bib}
\addbibresource{bibliographies/bib3.bib}
\addbibresource{bibliographies/bib4.bib}
\addbibresource{bibliographies/bib5.bib}
\addbibresource{bibliographies/bib6.bib}
\addbibresource{bibliographies/bib9.bib}

\begin{document}
\begin{center}
    \Huge \textbf{Building a quantum computer}
\end{center}
\vspace{-1em}

\begin{center}
    \emph{\large All of us}
\end{center}
\vspace{0.5em}
Lengths:\\ \\
Intro + background + Quantum Theory: 10-12 pages\\
Trapped Ions: 15 Pages\\
Photonic Quantum Computers: 10 pages\\
Conclusion: 1-2 Pages\\

\section{Introduction}
\lipsum[1-2]
\section{The Theory of nuclear forces}
Ever since it was discovered that the nucleus is made up of protons and neutrons, scientists have been working out how it is held together.
The logical answer would be a force, though the only two forces known at the time were the gravitational and the electromagnetic.
However, neither of these could be the one as protons and neutrons don't have opposite charges and the gravitational force is far too weak to explain the phenomenon.
% So it was concluded it must be another force, called the nuclear force, which we now know to be residual effects of the strong force between quarks, another fundamental force.
So it was concluded it must be another force, named the nuclear force.

\section{Mass Spectrometry}
A Mass Spectrometer (MS) is a device used to measure the mass to charge ratio ($\frac{m}{Q}$) of ions.
Usually, this charge to mass ratio is calculated from either how much an ion is deflected in a magnetic field or accelerated in an electric field – ions with the same ratio are deflected/accelerated equally.
This information can be used for a variety of purposes such as determining an unknown isotope in a mixture or to directly measure nuclear masses.

\section{Michael's Section}
Finding a physical representation of qubits that fulfil all the mathematical requirements is not easy.
Gaining control over a single quantum system has been a want since about 1970. word this so better
It is possible to see quantum effects on a vast number of combined quantum systems such as in superconductors. \cite{nielsen_quantum_2010}
Or in particle collisions again quantum effects are observed however there isn't control over a single quantum system.
Qubits are the physical realisation of controlling a single quantum system and its state.
You require a system with many degrees of freedom that can encode quantum information (qubits). \cite{bergou_quantum_2021}

\subsection{Optical Photon Quibit}

Why Photons?
\begin{itemize}
    \item Photons are massless, chargeless and don't interact with each other or other mass much. \cite{nielsen_quantum_2010}
    \item Can be guided long distances without much energy loss using optical fibers. \cite{nielsen_quantum_2010}
    \item Quantum information can be transmitted over long distances using photons as demonstrated by quantum entanglement over 1200km. \cite{yin_satellite-based_2017}
    \item photons can maintain entanglement over long distances and time period allowing the transmission quantum information. - this supper secure wow \cite{thibault_team_nodate}
\end{itemize}


How photons?

\begin{itemize}
    \item The transverse polarisation state of a photon can be used to represent qubits. 
    These are vertical, horizontal linear and left, right circular polarization. \cite{bergou_quantum_2021}
    \item These states can also be maintained - in isotropic materials photons polarization does not change as it propagates. \cite{bergou_quantum_2021}
    \item $\left\langle 0 \right\vert $
\end{itemize}
\section{Jans notes}
Ok, I'm taking over this section for now
\begin{itemize}

\item
Found this thing called DiVincezo criteria which might be nice to mention or something, it's 5 criteria that need to be met for a given hypothetical quantuum computer implementation:
\begin{itemize}
    \item A physical system containing well-defined 2 level quantum systems (qubits) which can be isolated from the environment.
    \item The ability to initialize this system in a well defined, determinate state.
    \item A set of universtal quantum gates which can be applied to each qubit or possibly pairs (or more) of them.
    \item Qubit decoherence times much greater than times for quantum gates to be applied.
    \item The ability to read out qubit state with high accuracy.
    \item ** In addition 2 more were mentioned by him, the ability to convert between "stationary" (like trapped ions) and "flying" (photons) states which would be necessary for a quantum network.
\end{itemize}

\item
    The Nielsen book lists quite a few of physical implementations for quantum computers - QHO, optical photon, optical cavity, ion traps and nuclear magnetic resonance

\item Trapped ion systems, kinda taken from Bruzewicz 2019, they also mention the DiVincezo stuff
\begin{itemize}
    \item All the main parts have been demonstrated, the problem is scaling.
    \item The actual states are the electronic states of some atom/ion.
    \item A couple types: hyperfine, Zeeman, fine structure and optical the differences between them is the particular energy levels that are used for states.
            The names make it decently clear, fine structure means the fine structure split states are used, optical means states which are transitioned between by photon emission/absorption.
    \item One of the things to deal with is that the atoms energies come from both their structure and their motion within the trap (which is also quantized).
    \item Some of the manipulations take place through lasers.
    \item Main pros are long coherence time and great gate and initialization/readout fidelity, drawbacks include speed, the gates are quite slow compared to other qubit technologies.
    \item Unordered random facts
    \begin{itemize}
        \item single qubit gates take a couple micro seconds and double about 10-100 micro seconds
        \item coherence times range from 0.2 to 600s (600s achieved for hyperfine) and surface code error correction was mentioned, the gate fielity should be good enough for that
        \item with respect to the optional DiVincezo, ions are unlikely to be the "flying" ones even though they can be moved, however high-fielity entanglement between ions and photons has been showed.
        \item most of the main things have been demonstrated by 2004 though the maximum number of ions achieved in a register so far was 20.
    \end{itemize}

    \item As far as I understand it is in fact a single ion that would represent a given qubit, the obvious way to store them are Penning or Paul traps.
    \item Paul traps rely on time dependent quadrupole fields (oscillating, usually trig or square waveforms).
    \item Penning traps do a mix of E and B fields to get a fairly complicated but alright motion.
    \item As far as I can tell Paul traps are considered the easier ones (and are more developed) but Penning traps have their advantages.

    \item Paul traps
    \begin{itemize}
        \item There are in fact 2 types.
        \item Point Paul traps have oscillating traps in all 3 dimensions, resulting in a single point with net 0 RF force (over a period).
            This can lead to issues when trapping multiple ions in the same trap as they tend to interact/overlap a lot.
        \item Linear Paul traps are the other type, they have an RF quadrupole in 2 dimensions and a static electric potential across the last one, this means the atoms can space better along that line according to that static potential.
        \item Okay, it's actually kinda insanely cool, so nowadays instead of it actually being 4 rods for the quadrupole field and some segments for the static field, they can deform this onto a surface -- as in the trap itself is a little plate with conducting segments to which the voltages are applied.
            Apparently it's still a good enough field and it is easy to make, there is plenty access for lasers and so on.
    \end{itemize}
    \item Atoms are loaded using resonant photo-ionization to make sure only a given isotope is present (in the past bombarding the atom gas with electrons as used and resulted in different isotopes being in the trap)
    \item Typically the excite states are quite high energy and multiple steps might be needed (often using near UV photons) to get them there.
    \item Apparently mostly elements from the 2nd group in the periodic table (or ones with similar properties) are used for trapped ion qubits.
    \item Any 2 states with long lifetimes can be used for the 0 and 1 states and almost always one of them is from the "ground state manifold" and based on where the other is the energy splittings can be anything between MHz to hundreds of THz.
    \begin{itemize}
        \item A big aspect of linear paul trap based systems is how it results in essentially a 1D crystal with the ions all trapped in the same harmonic potential along the axis, this results in shared motional states
        \item As we've just done in ICMP, there can be vibrational modes aka phonons which are "shared" among the ions (sorry don't have a simple but more formal explanation for now).
        \item There are 3N vibrational modes for this sort of a crystal with N ions and each mode is in a harmonic oscillator state (as in can be in a superposition of different harmonic oscillator states).
        \item The article I'm readidng now mentions that "lasers can be used to excite the internal electronic levels dependent upon the ions' vibrational states" which I'm not sure how but I see how that would lead to a way to entangle their states.
    \end{itemize}

    \item So I now skipped to more of the overview/applicability part.
    \item We might want to mention NISQ (Noisy Itentermeiate-Scale Quantum computing) -- it is a term describing the type of quantum computer we might be able to build now or in the very near future.
        It is essentially a subfield, studying what useful applications they could have is done. They require about 100s of qubits and as far as I can tell they "do not handle error correction" by themselves, the algorithms need to take care of that.
    \item A 53 qubit trapped ion system was used for further physics research (as far as I can understand they used it to simulate some complicate quantum many body system, they call is a quantum simulator).
    \item Quantum simulations using quantum computers seems to be a big thing in general, the Nature Physics 2012 Insight collections intro mentions atomic quantum gases, trapped ions, photonic systems and superconducting circuits being demonstrated.
    \item And I have seen a couple mentions of ~10s (up to 20 or so) qubit trapped ion systems though I didn't track them down right now.
\end{itemize}

\section{Trapped Ion Quantum Computing}
In this section we provide an introduction to the implementation details of trapped ion quantum computers (**once done possibly elaborate what exactly we covered).
The qubits in a trapped ion system are represented by individual trapped ions, as would have been seen in a quantum mechanics class, electrons bound in atoms can occupy a certain number of possible quantum states, here 2 of these level are chosen and used as the \kz and \ko states.
Though it isn't quite as simple as that, for quantum computing we must have a way of entangling these 2 states.
To achieve that, the qubits are trapped in the same trap (this will be explained further later on) and their motion within it is coupled as a quantum system through which entanglement is achieved.
Finally for qubit manipulation, lasers are used to excite and cool down the electrons and the ions themselves.

\subsection{Ion Trapping and the Paul Trap}
Ion trapping is an extensive field of expertise used for many purposes across physics and other sciences, it is an essential component of many experiments that try to manipulate individual particles, molecules or so on.
For a brief sketch of what this involves, these experiments must be done in vacuum (otherwise there would be too many other atoms around) and rely on complex electric and magnetic fields to manipulate the motion of charged atoms.
For quantum computing we are interested in trapping individual ions in a stable way, and while we want to trap multiple of them (currently about 5-100 has been achieved !43) it is important that we can tell them apart (as opposed to trapping them in bunches as is often done).

There are currently 2 dominant suitable ion trap designs, the Penning trap which uses a combination of magnetic and electric fields, and the Paul trap which uses time varying electric fields.
Penning traps are able to hold larger amounts of ions and more stably (300 ion crystals have been achieved !46), however the motion of the ions themselves is much more complicated and leads to qubit manipulation being harder to perform.
Because of that Paul traps are more commonly used for quantum computers and due to the scope of this report we will mainly focus on them.

\subsubsection{Simple Paul Trap}
Paul trap designs use time varying fields as it is not possible to confine an ion in 3D space with only static fields.
The simplest Paul trap design is composed of 4 conducting rods run in parallel to create a quadrupole electric field in the middle.
Then have each of the diagonally opposite rods connected at the same voltage and have these 2 voltages be some periodic function (usually a sine or a square wave) in antiphase, so that if at some time one is set to positive voltage, the other should be at negative voltage (**hopefully add a diagram/picture).
This results in a net effect of trapping a charged particle (within some range of mass to charge ratios, this setup is also called the Radio-Frequency-Quadrupole and can be used as a mass filter) along the axis of the quadrupole, why exactly this is well explained in (**add a source here).
Finally, for trapping along the last axis a static electric field is used, this can be achieved by for example 2 more rod segments along the quadrupole axis at either end of the whole setup, both at some positive voltage, resulting in a 


\subsubsection{Notes on Advanced Paul Trap designs}


% There are many different ways of constructing a Paul trap, different electrode geometries give different benefits and downsides.

% For quantum computing we are mainly interested to trap individual ions in a predictable and stable way, there are 2 main types of ions traps that are used, the Penning tr



\end{itemize}


\section{Notes on Entanglement and error correction}
\subsection{Entanglement}
\subsubsection{What is Quantum Entanglement (basic idea)?}

\subsubsection{Local Realism}
Local Realism is the combined principle of Locality and Realism.
\vspace{1em}

\textbf{What is Locality}

Locality is the principle which essentially states that objects can only be acted upon by the 'local' (or surrounding) space around it. So by locality, a particle at some far distance can only be acted upon by us if something such as an EM field travels through the medium between us to get to it.
\vspace{1em}

\textbf{What is Realism}

This principle refers to the existence of a pre-existing property that has a definite value within a particle, wave-function or object before you measure said property and can determine what value that is. It can be likened to the classic 'tree in an abandoned forest' thought experiment where realism assumes that the tree always exists in a specific state of upright or fallen despite your lack of knowledge of that state.
\vspace{1em}

\subsubsection{Bell States \& the Bell Inequality}
The reason why local realism is important is because the famous EPR paper which argues that our current understanding of quantum mechanics is incomplete (due to the phenomenon of quantum entanglement) assumes that local realism must be observed which John Bell in 1964 goes on to prove is an assumption not upheld in quantum mechanics. \\

The CHSH inequality is as follows:
\begin{itemize}
    \item Victor prepares 2 particles for Alice and Bob who are at some large distance from each other(large enough that they cannot communicate before the particles reach them), each will get one of the two particles
    \item Alice can decide to measure one of two properties on her particle with the measurements themselves being $A_{0}$ or $A_{1}$. These properties, once measured, can hold the values $a_{0}$ and $a_{1}$ respectively; both of these values can be $\pm 1$.  
    \item Bob can do the same with his measurements being denoted as $B_{0}$and $B_{1}$ and the values they can take being $b_{0}$ and $b_{1}$ respectively. And just the same $b_{0,1} = \pm 1$
    \item We're now going to perform a somewhat arbitrary sum where we go over all the possible combinations of values that can be measured: 
    \begin{equation}
    a_{0}b_{0} + a_{0}b_{1} + a_{1}b_{0} - a_{1}b_{1}
    \end{equation}
    This can be factorised to: 
    \begin{equation} \label{CHSH classical sum}
        (a_{0} + a_{1})b_{0} + (a_{0} - a_{1})b_{1}
    \end{equation}
    \item Since $a_{0}$ and $a_{1}$ can only = $\pm 1$ then $a_{0} = a_{1}$ or $a_{0} = -a_{1}$ so then looking at (\ref{CHSH classical sum}) we can see that either the $b_{0}$ term will vanish or the $b_{1}$ will
    \item The important takeaway from this is that (\ref{CHSH classical sum}) = $\pm 2$ and now we're going to calculate the average value of (\ref{CHSH classical sum}) by having Victor send many particles (two at a time, one to each of them, per trial) and having Alice and Bob measure each of these.
    \item Due to the fact that Alice and Bob can only perform one measurement at a time, (\ref{CHSH classical sum}) cannot be deduced in one singular trial but we'll assume that the underlying properties exist and then we'll take an average of each term to determine the average value of (\ref{CHSH classical sum}) over many trials
    \item This then gives what we call a Bell inequality:
    \begin{equation} \label{CHSH inequality}
        \langle A_{0}B_{0} \rangle + \langle A_{0}B_{1} \rangle + \langle A_{1}B_{0} \rangle- \langle A_{1}B_{1}\rangle \leq 2
    \end{equation}
\end{itemize}

The Bell inequality derived above shows that assuming locality (that Alice and Bob cannot communicate to each other through super-luminal methods) and realism (that the properties $a_{0}$, $a_{1}$, $b_{0}$ and $b_{1}$ exist without having to apply a measurement \textit{or measurement operator} such that you can take the averages of $A_{0}$, $A_{1}$, $B_{0}$ and $B_{1}$) without actually applying them to their respective particles) there is an upper bound with which the average of (\ref{CHSH classical sum}) can take and that is 2.

\vspace{1em}

We will show that taking Quantum Mechanics into account, we can reach an upper bound of $2\sqrt{2}$ which means that one (or both) of the two assumptions of Locality and Realism is not observed in Quantum Mechanics. This implies that Quantum Entanglement is not the result of some incomplete part of Quantum Mechanics (as suggested in the EPR paper) but rather a phenomena that just doesn't observe local realism.

Proof Using Bell States and Pauli Gates:
\begin{itemize}
    \item Victor send a pair of qubits to Alice and Bob and prepares them in the Bell State (\ref{Bell state Definition}).....
\end{itemize}

\subsubsection{How to perform Quantum Entanglement on a QC}
There are many ways to perform QE in a quantum circuit but the easiest and simplest methods create the maximally\footnote{When we say 'maximally entangled' we mean that the entangled state can be written as a sum of pure states and a pure state is one in which we have exact information about (such as $|0\rangle$). Beware of the fact that whilst this does mean we have information about the entangled state as a whole, we cannot say the same about the individual qubits/subsystems/bases that make that entangled state.} entangled states known as bell states \label{Bell state Definition}. One such way involves taking a 2-qubit circuit or state and performing a Hadamard Gate Operator on the first qubit, proceeding this you perform a Controlled-Not Gate Operator across the 2 qubits with the control being the first qubit and the target being the second. This then produces the Bell state $\frac{1}{\sqrt{2}}(|00\rangle + |11\rangle)$

%Matrix operation Math




\subsubsection{Using Quantum Entanglement}
\begin{itemize}
    \item Superdense coding
    \item Quantum Teleportation
    \item Quantum Key Distribution
\end{itemize}



\subsection{Error Correction}
\begin{itemize}
    \item Bit and Sign Flip
    \item Shor Code
    \item Toric Codes
\end{itemize}
\subsection{Notes on Topological Quantum Computers}
Several pairs of non-abelian anyons are created. These anyons can be compared to Majorna Zero Modes (MZM) (which are constructs that allow fermions to be described as two seperate halves). Individual anyons when locally observed are indistinguishable in their ground states but when you look at the entire system and swap two anyon positions then the wave function that describes the system changes. The action of swapping two anyons can be described by a 'braid' matrix that transforms the wave function. The idea of braiding anyons is important because unlike Bosons and Fermions whose wave functions remain the exact same after you swap two of them around each other twice, the anyon wave function keeps a count of how many times they were swapped (or rather, braided)
A QC can be built up by creating pairs of anyons (initialising your qubits), treating the braid matrices as different gate operators (depending on which anyons are swapped) and then bringing pairs of anyons together (effectively taking a measurement) and seeing which anyons annihilate completely and which ones release a fermion (the fermion produced from the two halves of the MZM.\\
\textit{Important note: Abelian anyons were first detected in 2020, Non-abelian anyons are yet to be detected but are an active area of experiment and research}

\section{Pablo's notes}

To effectively perform quantum computation it is necessary to be able to manipulate qubits and perform unitary operations on them. Furthermore, any unitary transformation can be composed of single qubit operations and CNOT gates. Thus, an experimental quantum computer should be able to implement them appropriately.

\subsection{Optical Photon Quantum Computer}

The correct combination of phase shifters and beam splitters allow the creation of any single qubit gate. This is a consequence of the theorem that states that any single qubit operation can be generated from z and y-axis rotations. A phase shifter performs $R_{z}$ rotations and a beamsplitter performs $R_{y}$ rotations.

Nonlinear Kerr media allow for the creation of a two qubit gate. The main experimental problem with this setup is succesfully making two photons interact. The available nonlinear Kerr media today cannot reliably obtain the $\pi$ cross phase modulation necessary to implement a CNOT gate. 

\subsection{Optical cavity quantum electrodynamics}

The main idea behind this model is the question of whether the state of a photon can be transferred to and from single atoms, whose interactions would be easier to control. Single qubit operations are constructed in the same way as in the optical photon quantum computer.

The CNOT gate can be implemented by coupling atoms enclosed in a Fabry-Perot cavity to the optical field of the photons.

\subsection{Ion Trap}

Have to talk it over. Have not managed to fully understand how quantum gates would be constructed in this setup.

\subsection{Nuclear Magnetic Resonance}

The main difference with this method compared to the previous is that we are acting on an ensemble of particles. Arbitrary single qubit transforms can be constructed from magnetic field pulses applied to spins in a strong magnetic field. Coupling between the molecules and, thus, the realization of a two qubit gate can be provided by chemical bonds between neighbouring atoms.

\section{Matthew's notes}

\subsection{What are registers?}
Registers are a form of memory for data instantaneously in use by the CPU (https://www.javatpoint.com/computer-registers). Any data that the CPU wants to process must first be stored in a register. Fundamentally, a register is a group of flip-flops which can each store a bit of data. 

Give an example of a register..
Explain what a flip flop is..

\subsection{Quantum Registers}
Quantum registers are analogous to classical registers because they store the data that is processed by the computer. 

Rather than a group of flip-flop circuits, it is a superposition of qubits. A classical register can store one of $2^n$ different values for $n$ flip-flops. By comparison, a quantum register of $n$ qubits can store all $2^n$ values simultaneously (https://cds.cern.ch/record/383367/files/p165.pdf). A quantum computer will require multiple registers for true computation (https://arxiv.org/pdf/quant-ph/9802065), for example by adding the contents of two registers together.

\subsection{Why use QCs?}
Originally proposed by Richard Feynman to solve quantum mechanics problems, these computers will be useful for molecular simulation which can be used for drug development. Particularly good for financial calculations which involve a lot of combinatorics (https://epjquantumtechnology.springeropen.com/articles/10.1140/epjqt/s40507-021-00091-1). Financial institutions will also be able to afford high investment into QC development. Typical programs use Monte Carlo simulation of market movements. Goldman Sachs is one example

\begin{itemize}

\item Explain why combinatorics problems are easily solved
\item Could talk more about Monte Carlo
\item Weather forecasting
\item https://learn.microsoft.com/en-us/azure/quantum/concepts-overview
\item https://medium.com/@markus.c.braun/a-brief-history-of-quantum-computing-a5babea5d0bd
\end{itemize}
Quantum computers have the potential to break much of today's encryption, particularly RSA which is based on prime factorisation. However, they also open the door to modern cryptography, such as quantum key distribution (QKD) which is theoretically unbreakable by laws of Physics rather than just mathematically difficult to solve.

\subsection{Why not to use QCs}
Although quantum computers have the potential to allow for many of the breakthroughs described above, they will not replace classical ones. Instead, both systems are likely to coexist. Many of the things that an average user uses a computer for could not be enhanced to for any practical benefit by a quantum computer. There is some classical computation that could in fact be much slower on a quantum machine due to all the extra overheads required to run one. Streaming video or writing documents involves a certainty in the data, if you press a key on the keyboard there is no ambiguity in what to store that as internally. Therefore, there is no need to consider all possibilities when doing day to day computation (https://ieeexplore.ieee.org/stamp/stamp.jsp?tp=&arnumber=8322045).

\subsection{Major Breakthroughs}
\begin{itemize}
    \item 1998 - first demonstration of two qubit system (https://semanticscholar.org/paper/6c055053f4f1605fdc0bd474c7a350dcd01f627d)
    \item 2019 - google declares quantum supremacy by performing a series of operations in 200 seconds that would take a supercomputer about 10,000 years to complete; IBM responds by suggesting it could take 2.5 days instead of 10,000 years, highlighting techniques a supercomputer may use to maximize computing speed (https://www.forbes.com/sites/gilpress/2021/05/18/27-milestones-in-the-history-of-quantum-computing/?sh=2506dc67b23f)
    \item 2021, IBM eagle which is 127 bit quantum processor. First QC which is so complex that it cannot be simulated reliably by a classical computer 'the number of classical bits necessary to represent a state on the 127-qubit processor exceeds the total number of atoms in the more than 7.5 billion people alive today' (https://newsroom.ibm.com/2021-11-16-IBM-Unveils-Breakthrough-127-Qubit-Quantum-Processor)
\end{itemize}


\section{Superconducting QC stuff}

\begin{itemize}
    \item Currently used by Google, IBM, etc. and other big companies
    \item Cooper pairs are formed when an electron moves through a lattice, it pulls positive ions in the lattice closer to it as it moves along, creating a 'ripple' effect of increasing positive charge distribution, which then attracts nearby electrons to the ripple and forms a couple with the first electron. These electrons are paired and are called Cooper pairs. Their mutual repulsion keeps them away from each other and propels one when the other gets too close, and the other is pulled to the first when the first one creates a denser positive charge distribution. This way they propel each other without force and form superconducting electrons. 
    \item 3 types – Flux, Charge, Phase
    \item Most helpful \url{https://pennylane.ai/qml/demos/tutorial_sc_qubits.html}
    \item DVC1 : Very scalable because Josephson junctions are made on chips with 'easy' manufacturing techniques, however since the principle is having quantum effects on macroscopic level, too much scaling has issues with losing some quantum properties.
    \item DVC2 : Easily satisfied, because "Since the excited states in an artificial atom are short-lived and prefer to be on the ground state, all we have to do is wait for a short period. If the circuit is well isolated, it is guaranteed that all the qubits will be in the ground state with a high probability after this short interval." from PennyLane
    \item DVC3 : Hard because qubits are short-lived
    \item DVC4 : Somewhat challenging, currently done using 'capacitative coupling', however WiKipedia Superconducting QC page says doable
    \item DVC5 : easily doable by shining specific frequency light that brings excited state to
    \item 
    \url{https://jonathan-hui.medium.com/qc-how-to-build-a-quantum-computer-with-superconducting-circuit-4c30b1b296cd}
    \item 
    \url{https://www.reddit.com/r/askscience/comments/h0t51/comment/c1rr48v/?utm_source=share&utm_medium=web2x&context=3}
    \item \url{https://quantum.phys.cmu.edu/QCQI/QC_CMU2}
    \item \url{https://www.nature.com/articles/s41586-019-1666-5}
    \item \url{https://www.nature.com/articles/nature07128#Sec1}
    \item \url{https://www.nature.com/articles/s41534-016-0004-0#Sec2}

\end{itemize}
\section{Linear Optical Quantum Computing}
%%%%%%%%%%%%%%%%%%%%%%%%%%%%%%%%%%%%%%%%%%

\subsection{Qubit representation}
\subsubsection{Polarisation modes}
Horizontal and Vertical Polarisation
\subsubsection{Spatial Modes}
Single Rail and Dual Rail Encoding


%%%%%%%%%%%%%%%%%%%%%%%%%%%%%%%%%%%%%%%%%%

\subsection{Useful Linear Optical Tools}
\subsubsection{Beam Splitters}
\subsubsection{Phase Shifters}
\subsubsection{Photo-detectors}

%%%%%%%%%%%%%%%%%%%%%%%%%%%%%%%%%%%%%%%%%%

\subsection{Single Qubit Operations}
As we have seen above, all other important single qubit operators can be constructed from the Pauli gates so all that we need to do is ensure we can that we have some physical representation for the Pauli gate operations.
\subsubsection{Pauli X Representation}
\subsubsection{Pauli Y Representation}
\subsubsection{Pauli Z Representation}

%%%%%%%%%%%%%%%%%%%%%%%%%%%%%%%%%%%%%%%%%%

\subsection{Efficiency}
As we know by now, the real magic of quantum computation occurs when we can implement 2 qubit gates as this allows us to perform quantum entanglement between qubit states which is then instrumental in many other constructs such as Quantum Teleportation, Quantum Algorithms, etc.

The issue we tend to run across with LOQC is that photons tend to not naturally interact with each other. One suggested method of overcoming this uses photo-detectors to make projective measurements which can induce an interaction between photons\cite{Kok:2005jip}. This is a probabilistic approach with the probability of success being rather small and the method for increasing this probability to "near-one" values scales exponentially with resources (optical modes required). This makes optical computation seem quite unfeasible on the large scale but thanks to a few interesting results in quantum optics we can implement what is referred to as the KLM protocol to construct scalable 2-qubit gates.


\subsubsection{Three results that allow for efficient LOQC}
Due to the following results the additional resources for a LOQC that scales with n, scales with O($n^2$):
\begin{itemize}
    \item Non-deterministic quantum computation is possible with linear optics
    \item The probability of success can be increased close to one
    \item The resources needed for "accurately" (probability of success close to 1) encoded qubits scales efficiently  
\end{itemize}

%%%%%%%%%%%%%%%%%%%%%%%%%%%%%%%%%%%%%%%%%%%

\subsection{Two Qubit Operations}
\subsubsection{Attempting Entanglement}

%%%%%%%%%%%%%%%%%%%%%%%%%%%%%%%%%%%%%%%%%%



\subsection{Fault tolerance}
\subsubsection{Photon loss}

\printbibliography

\end{document}
