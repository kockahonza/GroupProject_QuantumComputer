\documentclass[a4paper]{article}
\usepackage[top=60pt,bottom=80pt,left=60pt,right=60pt]{geometry}
\usepackage[utf8]{inputenc}
\usepackage[style=phys,url=false,doi=false,isbn=false,eprint=false]{biblatex}
\usepackage{amsmath}
\usepackage{cleveref}
\usepackage{physics}
\usepackage{siunitx}
\usepackage{lipsum}

\addbibresource{bibliographies/bib1.bib}
\addbibresource{bibliographies/bib2.bib}
\addbibresource{bibliographies/bib3.bib}
\addbibresource{bibliographies/bib4.bib}
\addbibresource{bibliographies/bib5.bib}
\addbibresource{bibliographies/bib6.bib}

% Macros for the 0 and 1 kets, might be a useful shorthand, if you want to use it just do \kz or \ko anywhere
\newcommand{\kz}{$\ket{0}$ }
\newcommand{\ko}{$\ket{1}$ }

\begin{document}
\begin{center}
    \Huge \textbf{Building a quantum computer}
\end{center}
\vspace{-1em}

\begin{center}
    \emph{\large All of us}
\end{center}
\vspace{0.5em}

\section{Introduction}
\begin{itemize}
    \item brief overview of what a quantum computer is
    \item brief timeline/history of quantum computers
    \item where does this report fit into current literature
\end{itemize}
\subsection{Why build a quantum computer?}
\begin{itemize}
    \item RSA encryption
    \item How QC is better at breaking RSA encryption (explain simultaneous states)
    \item Where else can the benefits be used (other than in cryptography, e.g. bio simulation)
\end{itemize}
\subsection{How to build a quantum computer}
This is basically just acknowledging there's different ways of doing it, mention where Google, IBM and IonQ are at. Then say we've chose Trapped Ion and explain structure of report 
\begin{itemize}
    \item multiple ways of doing it
    \item give some historical developments on the different technologies
    \item who are the key players in building quantum computers
    \item briefly explain how the report will be structured, why have we chosen it this way
\end{itemize}

The purpose of this report is to describe the physical realisation of a quantum computer. 
It will give a brief explanation of the quantum mechanical concepts which are key to quantum computing such as entanglement and superposition. 
The theory of storing and processing of information using quantum mechanical systems is described broadly before a detailed insight into `building' a quantum computer.
Quantum computing is a very active field of research with numerous methods of implementation being studied including photonic and superconducting systems. 

Quantum supremacy is a term used to describe when a quantum computer is able to complete a calculation that a classical computer could not achieve in a reasonable amount of time. 
Google was first to claim quantum supremacy in 2019 with their 53 bit `Sycamore' quantum computer which reportedly solved a problem in under 5 hours which would take a classical computer over 10 000 years. \cite{gibney_hello_2019} 
However these claims were disputed by IBM - another large company heavily invested in the quantum computing race.
The largest quantum computer to date is the IBM osprey with 433 qubits announced in November 2022, both the Osprey and Sycamore computers use superconducting qubits (where qubits are the quantum realisation of classical computer bits).

This report will focus on trapped ion systems where in [date] [company] announced a [nbit] computer. Although trapped ions have not yet reached the number of qubits seen from superconducting systems there are many advantages to them, some of these are: 
\begin{enumerate}
    \item The cost of ion trap systems is relatively low - therefore research can be carried out by academics and be peer-reviewed. This affects the amount of research materials available on the public domain. As opposed to superconducting systems which require vast investments leading to only a select few companies able to reseach it and they do not publish much reseach due to their stategic ad
\end{enumerate}
\section{The Theory of nuclear forces}
\lipsum[2-3]
\section{Mass Spectrometry}
\lipsum[3-4]
\section{Michael's Section}
Finding a physical representation of qubits that fulfil all the mathematical requirements is not easy.
Gaining control over a single quantum system has been a want since about 1970. word this so better
It is possible to see quantum effects on a vast number of combined quantum systems such as in superconductors. \cite{nielsen_quantum_2010}
Or in particle collisions again quantum effects are observed however there isn't control over a single quantum system.
Qubits are the physical realisation of controlling a single quantum system and its state.
You require a system with many degrees of freedom that can encode quantum information (qubits). \cite{bergou_quantum_2021}

\subsection{Optical Photon Quibit}

{\bf Why Photons?}
\begin{itemize}
    \item Photons are massless, chargeless and don't interact with each other or other mass much. \cite{nielsen_quantum_2010}
    \item Can be guided long distances without much energy loss using optical fibers. \cite{nielsen_quantum_2010}
    \item Quantum information can be transmitted over long distances using photons as demonstrated by quantum entanglement over 1200km. \cite{yin_satellite-based_2017}
    \item photons can maintain entanglement over long distances and time period allowing the transmission quantum information. - this supper secure wow \cite{thibault_team_nodate}
\end{itemize}

\vspace{1em}
{\bf How photons?}

\begin{itemize}
    \item The transverse polarisation state of a photon can be used to represent qubits. 
    These are vertical, horizontal linear and left, right circular polarization. \cite{bergou_quantum_2021}
    
    \item These states can also be maintained - in isotropic materials photons polarization does not change as it propagates. \cite{bergou_quantum_2021}
    
    \item $\left\vert 0 \right\rangle = \left\vert H\right\rangle $ and $\left\vert 1 \right\rangle = \left\vert V\right\rangle $ where H and V are horizontal and vertical polarisation. \cite{bergou_quantum_2021}

\end{itemize}
\vspace{1em}
{\bf Need a single photon source}
\begin{itemize}
    \item lasers with very low output can emit single electrons... but 90$\%$ of the time no photon is emitted and when there is a photon emitted 5$\%$ of than one is emitted.\cite{nielsen_quantum_2010}
    \begin{itemize}
        \item so lasers CANNOT be used - we need sources that can be synchronised not possible if you don't know when a photon is actual coming out
    \end{itemize}
\end{itemize}
\vspace{1em}
{\bf Quantum Dot}
\begin{itemize}
    \item Quantum dots can (kinda) be a single photon source
    \item from wiki quantum dots seem to be `semiconducting particles a few nanometers in size' therefor shining light on it would emit one photon is the idea
    \item need to understand $T_1$ and $T_2$ coherance times.... think $T_2$ is the time taken for the inital and final states to be uncorrelated (i.e. $|1\rangle \rightarrow \alpha |0\rangle + \beta |1\rangle$ a liner combination of them both?
    \item anyway... using a narrow-linewidth laser which had the same resonance as the quantum dot single photon emission was achieved with a $T_2=22ns$ (apparently long) was achieved. \cite{lodahl_interfacing_2015}
\end{itemize}
\vspace{1em}
{\bf DiVincenzo Criteria} \cite{bergou_quantum_2021}
\begin{itemize}
    \item Scalability with well defined qubits
    
    \begin{itemize}
        \item $\left\vert 0 \right\rangle = \left\vert H\right\rangle $ and $\left\vert 1 \right\rangle = \left\vert V\right\rangle $ where H and V are horizontal and vertical polarisation.  
        \item polarisation state can be maintained for a long time
        \item or timebin/path encoding \cite{obrien_optical_2007}

        \item eventually need like $10^{11}$ single photon states. posible way of doing this is using $10^11$ sources that can reliably produce a photon as in when requested
        \begin{itemize}
            \item don't exist yet, but are being worked on some promising are quantum dots, trapped ions and atoms, color centers in diamonds, semiconductors \cite{slussarenko_photonic_2019}
        \end{itemize}
    \end{itemize}

    \item Initializing qubits to a simple fiducial state
    \begin{itemize}
        \item quantum dot - single photon source. when excited on resonance by low laser prob of emmiting two photons instead of one is super low \cite{santori_indistinguishable_2002}
    \end{itemize}

    \item  A qubit-specific measurement capability
    \begin{itemize}
        \item an ideal photon detector which detects every individual photon with no dark count (false positive) doesn't yet exist. 
        \item one of the most used detectors currently is the Si avalanche photodiode... only has a detection efficiency of 65$\%$! so the probability of detecting 10 photons with 10 detectors is less than $2\%$. \cite{slussarenko_photonic_2019}
        \item better now we have superconducting nanowire single-photon detectors (SNSPDs) with reset rate $\sim$40ns and detection efficiency $>95\%$ ... but these need temos of around 2K. \cite{santori_indistinguishable_2002}
        
    \end{itemize}
    
    \item Long relevant decoherence times
    \begin{itemize}
        \item 
    \end{itemize}

    \item A ``universal'' set of quantum gates
    \begin{itemize}
        \item birefringent wave plates for one qubit gates \cite{obrien_optical_2007}
    
        \item deeterministic CNOT gates difficult (these are the double qubit ones). A 2007 paper suggests that for a near deterministic CNOT gate (i.e. not deterministic just low prob) it requires $>$10 000 entangled photons for only $>95\%$ success probability \cite{obrien_optical_2007}
    \end{itemize}

    \item The ability to interconvert stationary and flying qubits
    \begin{itemize}
        \item can use polarizing beam splitters to convert between polarisation and path encoding \cite{obrien_optical_2007}
    \end{itemize}

    \item The ability to transmit flying qubits between specified locations
    \begin{itemize}
        \item well photons are transmitted all the time!
        \item actually kind of difficult cause photons are always 'flying' so some kind of optical quantum memory may be necesarry to store them. 
    \end{itemize}
\end{itemize}

\section{Jans notes}
Ok, I'm taking over this section for now
\begin{itemize}

\item
For "types" fo QCters I see photonic and trapped ion for now at least.

\item
Trapped ion type seems quite nice, apparently it works well, but they haven't really scaled to hundreds of qubits.


\item
Found this thing called DiVincezo criteria which might be nice to mention or something, it's 5 criteria that need to be met for a given
hypothetical QCter type/implementation:
\begin{itemize}
    \item A physical system containing well-defined 2 level quantum systems (qubits) which can be isolated from the environment.
    \item The ability to initialize this system in a well defined, determinate state.
    \item A set of universtal quantum gates which can be applied to each qubit or possibly pairs (or more) of them.
    \item Qubit decoherence times much greater than times for quantum gates to be applied.
    \item The ability to read out qubit state with high accuracy.
    \item ** In addition 2 more were mentioned by him, the ability to convert between "stationary" (like trapped ions) and "flying" (photons) states which would be necessary for a quantum network.
\end{itemize}

\item Trapped ion systems, kinda taken from Bruzewicz 2019, they also mention the DiVincezo stuff
\begin{itemize}
    \item The actual states are the electronic states of some atom/ion.
    \item A couple types: hyperfine, Zeeman, fine structure and optical the differences between them is the particular energy levels that are used for states.
            The names make it decently clear, fine structure means the fine structure split states are used, optical means states which are transitioned between by photon emission/absorption.
    \item One of the things to deal with is that the atoms energies come from both their structure and their motion within the trap (which is also quantized).
    \item Some of the manipulations take place through lasers.
    \item Main pros are long coherence time and great gate and initialization/readout fidelity, drawbacks include speed, the gates are quite slow compared to other qubit technologies.
    \item Unordered random facts
    \begin{itemize}
        \item single qubit gates take a couple micro seconds and double about 10-100 micro seconds
        \item coherence times range from 0.2 to 600s (600s achieved for hyperfine) and surface code error correction was mentioned, the gate fielity should be good enough for that
        \item with respect to the optional DiVincezo, ions are unlikely to be the "flying" ones even though they can be moved, however high-fielity entanglement between ions and photons has been showed.
        \item most of the main things have been demonstrated by 2004 though the maximum number of ions achieved in a register so far was 20.
    \end{itemize}
\end{itemize}

\end{itemize}



\section{Summary}
Studies of the nuclear force are a fascinating yet unfinished part of physics.
Though most simple interactions involving it are well understood, it is yet to be applied to more complicated systems.
Mass spectrometers have been crucial in testing nuclear theories and pushing their limits.
They have allowed for the exploration of the nuclear landscape closer to the neutron dripline where models using only 2N interactions have been insufficient in predicting the shell closures of these exotic nuclei.
Future MS such as the PENTATRAP experiment or the many MR-TOF spectrometers that are being developed, with improved resolving power and shorter needed observation time will be crucial in probing and improving the leading nuclear force theories.

\section{Pablo's notes}

To effectively perform quantum computation it is necessary to be able to manipulate qubits and perform unitary operations on them. Furthermore, any unitary transformation can be composed of single qubit operations and CNOT gates. Thus, an experimental quantum computer should be able to implement them appropriately.

\subsection{Optical Photon Quantum Computer}

The correct combination of phase shifters and beam splitters allow the creation of any single qubit gate. This is a consequence of the theorem that states that any single qubit operation can be generated from z and y-axis rotations. A phase shifter performs $R_{z}$ rotations and a beamsplitter performs $R_{y}$ rotations.

Nonlinear Kerr media allow for the creation of a two qubit gate. The main experimental problem with this setup is succesfully making two photons interact. The available nonlinear Kerr media today cannot reliably obtain the $\pi$ cross phase modulation necessary to implement a CNOT gate. 

\subsection{Optical cavity quantum electrodynamics}

The main idea behind this model is the question of whether the state of a photon can be transferred to and from single atoms, whose interactions would be easier to control. Single qubit operations are constructed in the same way as in the optical photon quantum computer.

The CNOT gate can be implemented by coupling atoms enclosed in a Fabry-Perot cavity to the optical field of the photons.

\subsection{Ion Trap}

Have to talk it over. Have not managed to fully understand how quantum gates would be constructed in this setup.

\subsection{Nuclear Magnetic Resonance}

The main difference with this method compared to the previous is that we are acting on an ensemble of particles. Arbitrary single qubit transforms can be constructed from magnetic field pulses applied to spins in a strong magnetic field. Coupling between the molecules and, thus, the realization of a two qubit gate can be provided by chemical bonds between neighbouring atoms.

\section{Matthew's notes}

\subsection{What are registers?}
Registers are a form of memory for data instantaneously in use by the CPU (https://www.javatpoint.com/computer-registers). Any data that the CPU wants to process must first be stored in a register. Fundamentally, a register is a group of flip-flops which can each store a bit of data. 

Give an example of a register..
Explain what a flip flop is..

\subsection{Quantum Registers}
Quantum registers are analogous to classical registers because they store the data that is processed by the computer. 

Rather than a group of flip-flop circuits, it is a superposition of qubits. A classical register can store one of $2^n$ different values for $n$ flip-flops. By comparison, a quantum register of $n$ qubits can store all $2^n$ values simultaneously (https://cds.cern.ch/record/383367/files/p165.pdf). A quantum computer will require multiple registers for true computation (https://arxiv.org/pdf/quant-ph/9802065), for example by adding the contents of two registers together.

\subsection{Why use QCs?}
Originally proposed by Richard Feynman to solve quantum mechanics problems, these computers will be useful for molecular simulation which can be used for drug development. Particularly good for financial calculations which involve a lot of combinatorics (https://epjquantumtechnology.springeropen.com/articles/10.1140/epjqt/s40507-021-00091-1). Financial institutions will also be able to afford high investment into QC development. Typical programs use Monte Carlo simulation of market movements. Goldman Sachs is one example

\begin{itemize}

\item Explain why combinatorics problems are easily solved
\item Could talk more about Monte Carlo
\item Weather forecasting
\item https://learn.microsoft.com/en-us/azure/quantum/concepts-overview
\item https://medium.com/@markus.c.braun/a-brief-history-of-quantum-computing-a5babea5d0bd
\end{itemize}
Quantum computers have the potential to break much of today's encryption, particularly RSA which is based on prime factorisation. However, they also open the door to modern cryptography, such as quantum key distribution (QKD) which is theoretically unbreakable by laws of Physics rather than just mathematically difficult to solve.

\subsection{Why not to use QCs}
Although quantum computers have the potential to allow for many of the breakthroughs described above, they will not replace classical ones. Instead, both systems are likely to coexist. Many of the things that an average user uses a computer for could not be enhanced to for any practical benefit by a quantum computer. There is some classical computation that could in fact be much slower on a quantum machine due to all the extra overheads required to run one. Streaming video or writing documents involves a certainty in the data, if you press a key on the keyboard there is no ambiguity in what to store that as internally. Therefore, there is no need to consider all possibilities when doing day to day computation (https://ieeexplore.ieee.org/stamp/stamp.jsp?tp=&arnumber=8322045).

\subsection{Major Breakthroughs}
\begin{itemize}
    \item 1998 - first demonstration of two qubit system (https://semanticscholar.org/paper/6c055053f4f1605fdc0bd474c7a350dcd01f627d)
    \item 2019 - google declares quantum supremacy by performing a series of operations in 200 seconds that would take a supercomputer about 10,000 years to complete; IBM responds by suggesting it could take 2.5 days instead of 10,000 years, highlighting techniques a supercomputer may use to maximize computing speed (https://www.forbes.com/sites/gilpress/2021/05/18/27-milestones-in-the-history-of-quantum-computing/?sh=2506dc67b23f)
    \item 2021, IBM eagle which is 127 bit quantum processor. First QC which is so complex that it cannot be simulated reliably by a classical computer 'the number of classical bits necessary to represent a state on the 127-qubit processor exceeds the total number of atoms in the more than 7.5 billion people alive today' (https://newsroom.ibm.com/2021-11-16-IBM-Unveils-Breakthrough-127-Qubit-Quantum-Processor)
\end{itemize}


\printbibliography

\end{document}
